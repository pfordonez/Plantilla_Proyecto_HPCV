% Plantilla LaTeX para Overleaf
% Guía integrada de análisis del proyecto final

\documentclass[12pt]{article}

% Idioma y codificación
\usepackage[spanish]{babel}
\usepackage[utf8]{inputenc}
\usepackage[T1]{fontenc}

% Márgenes y formato general
\usepackage{geometry}
\geometry{margin=2.5cm}

% Listas y tablas
\usepackage{enumitem}
\usepackage{array}

% Hipervínculos
\usepackage{hyperref}
\hypersetup{
  colorlinks=true,
  linkcolor=blue,
  urlcolor=blue
}

% Título del documento (puede ajustarse según la asignatura)
\title{Guía Integrada para el Análisis del Proyecto Final\\ \large Asignatura: Human Perception \\ Computer Vision}
\author{Docente: Pablo F. Ordoñez Ordoñez.}
\date{}

\begin{document}

\maketitle

\section*{Propósito de la guía}

Esta guía unifica y adapta los lineamientos de las tareas de proyecto para la asignatura \textbf{Human Perception \\ Computer Vision}, con el fin de orientar a los estudiantes en la elaboración del \textbf{análisis de su proyecto final}. 

Cada proyecto debe considerar explícitamente \textbf{tres dimensiones obligatorias}:

\begin{enumerate}[label=\textbf{\arabic*.}]
  \item \textbf{Human Perception (HP):} procesos de percepción visual humana, atención, memoria, carga cognitiva, sesgos y límites sensoriales, FATE.
  \item \textbf{Computer Vision (CV):} adquisición, preprocesamiento, representación y análisis automatizado de información visual.
  \item \textbf{Técnicas de IA para CV y ML/Deep Learning (AI/ML/DL):} modelos y algoritmos de inteligencia artificial que permiten aprender patrones complejos (por ejemplo: CNN, RNN, transformers, modelos de clasificación, segmentación o detección para visión por computador).
\end{enumerate}

El documento de análisis de proyecto final deberá seguir la estructura que se detalla a continuación.

\section{Título del proyecto}

El título debe reflejar claramente el propósito del proyecto e idealmente evidenciar la integración de al menos dos de las tres dimensiones (HP, CV, AI/ML/DL).

\vspace{0.2cm}
\noindent
\textbf{Ejemplo:} \\ 
\emph{"Sistema para la detección de fatiga visual en estudiantes utilizando principios de percepción humana, visión por computador y redes neuronales profundas"}.

\section{Datos del estudiante}

\begin{itemize}
  \item Nombre completo del estudiante.
  \item Correo institucional.
  \item Carrera y ciclo.
\end{itemize}

\section{Definición del problema (versión computacional)}

En esta sección se debe describir el problema de forma clara, sin mezclar aún detalles de la solución. Considere:

\begin{enumerate}[label=\textbf{\alph*)}]
  \item \textbf{Contexto general del problema:} ¿En qué escenario surge? (educativo, industrial, médico, urbano, etc.).
  \item \textbf{Dimensión de Human Perception (HP):} 
  \begin{itemize}
    \item ¿Qué procesos de percepción humana están involucrados? (atención, percepción del movimiento, reconocimiento de objetos, emociones, carga cognitiva, FATE, etc.).
    \item ¿Qué limitaciones o sesgos perceptuales influyen en el problema? (fatiga, tiempo de reacción, campo visual reducido, sobrecarga de estímulos, etc.).
  \end{itemize}
  \item \textbf{Dimensión de Computer Vision (CV):}
  \begin{itemize}
    \item ¿Qué tipo de datos visuales o señales sensoriales se requieren? (imágenes, video, secuencias de profundidad, flujo óptico, etc.).
    \item ¿Qué operaciones básicas de visión por computador son necesarias? (detección, segmentación, seguimiento de objetos, extracción de características, etc.).
  \end{itemize}
  \item \textbf{Dimensión de AI/ML/DL aplicada a CV:}
  \begin{itemize}
    \item ¿Por qué el problema requiere técnicas de inteligencia artificial o aprendizaje profundo? 
    \item ¿Qué patrones o relaciones en los datos no se podrían resolver de forma trivial sin un modelo de aprendizaje? (clasificación de estados, reconocimiento de gestos, predicción de eventos, etc.).
  \end{itemize}
  \item \textbf{Justificación de relevancia:}
  \begin{itemize}
    \item ¿Por qué este problema es importante a nivel local (idealmente en Ecuador o Latinoamérica)?
    \item ¿Qué impacto tendría una solución efectiva basada en HP + CV + AI/ML/DL?
  \end{itemize}
\end{enumerate}

\section{Objetivo general y objetivos específicos}

Formule un \textbf{objetivo general} centrado en el problema y en la integración de las dimensiones. Posteriormente, liste los \textbf{objetivos específicos}.

\subsection*{Ejemplos de formulación}

\begin{itemize}
  \item \textbf{Objetivo general:} \\ Desarrollar un sistema que estime el nivel de fatiga visual en estudiantes durante sesiones de estudio, integrando principios de percepción humana, visión por computador y redes neuronales profundas.
  \item \textbf{Objetivos específicos:}
  \begin{itemize}
    \item Desarrollar una interfaz gráfica interactiva que permita registrar, presentar estímulos visuales y recopilar respuestas del usuario para evaluar procesos de percepción humana (atención, reconocimiento, tiempo de reacción o carga cognitiva).
    \item Implementar un módulo de visión por computador combinado con técnicas de IA/ML/DL para analizar las respuestas visuales o comportamentales registradas en la interfaz, con el fin de clasificar, estimar o predecir indicadores perceptuales (por ejemplo: atención, fatiga, reconocimiento, precisión visual).
    
  \end{itemize}
\end{itemize}




\section{Usuarios o clientes potenciales}

Describa quiénes se beneficiarían de la solución:

\begin{itemize}
  \item Organizaciones o instituciones (en Ecuador o Latinoamérica) donde el problema es relevante.
  \item Perfiles de usuario final: estudiantes, docentes, operadores, técnicos, personal médico, etc.
  \item Para cada tipo de usuario, señale brevemente cómo se relaciona con HP, CV y AI/ML/DL.
\end{itemize}

\section{Modelos de análisis del problema}

En esta sección se deben incluir modelos de análisis utilizando un lenguaje de modelado formal (por ejemplo, UML u otro lenguaje ingenieril). \textbf{Estos modelos describen solo el problema, no la solución detallada.}

\subsection{Modelos mínimos sugeridos}

\begin{enumerate}[label=\textbf{\arabic*.}]
  \item \textbf{Modelo de contexto del sistema:} representa el entorno donde ocurre el problema y los actores principales (usuarios, sensores, sistemas externos).
  \item \textbf{Casos de uso centrados en HP y CV:}
  \begin{itemize}
    \item Interacciones del usuario con el sistema desde la perspectiva de percepción humana (ej.: observa estímulos, responde, se fatiga, se distrae).
    \item Interacciones relacionadas con la captura y procesamiento de información visual (ej.: cámara registra rostro, sistema analiza movimiento ocular).
    \item 2 escenarios por cada caso de uso
  \end{itemize}
  \item \textbf{Modelo conceptual (clases o entidades):}
  \begin{itemize}
    \item Entidades como: \emph{Usuario}, \emph{Sesión}, \emph{Imagen}, \emph{Video}, \emph{Señal de entrada}, \emph{Etiqueta perceptual}, \emph{Métrica de desempeño}, entre otras.
    \item Atributos y relaciones relacionadas con percepciones humanas, datos visuales y anotaciones para entrenamiento de modelos.
  \end{itemize}
  \item \textbf{Diagramas de actividades o flujo del fenómeno perceptual:}
  \begin{itemize}
    \item Secuencia de acciones que sigue el usuario y el sistema.
    \item Puntos donde la percepción humana se ve afectada y donde entra el análisis de CV.
  \end{itemize}
\end{enumerate}

\subsection{Resumen de dimensiones en los modelos}

\begin{center}
\begin{tabular}{|>{\raggedright}p{3.2cm}|>{\raggedright}p{10cm}|}
\hline
\textbf{Dimensión} & \textbf{Qué modelar} \\
\hline
Human Perception (HP) & Estímulos, atención, percepción del usuario, emociones, fatiga, carga cognitiva, sesgos, tiempos de reacción. \\
\hline
Computer Vision (CV) & Flujo visual, cámara o sensor, preprocesamiento (filtros, normalización), detección, segmentación, extracción de rasgos. \\
\hline
AI/ML/DL para CV & Tipos de datos de entrada y salida, etiquetas, características que requieren aprendizaje, tareas de clasificación, regresión, segmentación o detección. \\
\hline
\end{tabular}
\end{center}

\section{Descripción preliminar de la solución}

En esta sección se introduce una primera versión conceptual de la solución (sin entrar todavía en detalles de implementación):

\begin{itemize}
  \item Cómo se integran HP, CV y AI/ML/DL en el sistema propuesto.
  \item Tipo de técnicas de IA para visión que se podrían usar (por ejemplo: CNN, redes siamesas, modelos de segmentación semántica, modelos de detección de objetos, modelos de aprendizaje por refuerzo profundo, etc.).
  \item Breve descripción del conjunto de datos:
  \begin{itemize}
    \item Fuente de los datos (propios, públicos, institucionales).
    \item País u origen (preferible Ecuador o contexto latinoamericano).
    \item Método de recolección (sensores, cámaras, plataformas, encuestas, etc.).
    \item Un ejemplo de la estructura de los datos (columnas o tipos de archivos).
  \end{itemize}
\end{itemize}

\section{Implementación (plan inicial)}

Describa de forma general cómo planea implementar la solución:

\begin{itemize}
  \item Lenguaje(s) de programación (por ejemplo, Python, C++, etc.).
  \item Entorno o framework principal (por ejemplo, OpenCV, TensorFlow, PyTorch, Keras, Unity con ML-Agents, etc.).
  \item Librerías específicas para visión por computador y aprendizaje profundo.
  \item Requerimientos de hardware (cámara, GPU, dispositivos de captura, entornos inmersivos, etc.).
\end{itemize}

\section{Resultados esperados y evaluación}

Explique qué espera obtener y cómo evaluará la solución:

\begin{itemize}
  \item Resultados cualitativos (mejora de la experiencia perceptual, reducción de errores humanos, apoyo al aprendizaje, etc.).
  \item Resultados cuantitativos (métricas de desempeño del modelo de IA/ML/DL).
  \item Métricas posibles: exactitud (accuracy), precisión, recall, F1-score, AUC-ROC, IoU (para segmentación), etc.
  \item Cómo evaluará el componente de percepción humana (cuestionarios, estudios de usuario, tiempos de reacción, etc.).
  \item Cómo evaluará el componente de visión por computador (robustez ante ruido, cambios de iluminación, diferentes usuarios, etc.).
\end{itemize}

\section{Referencias}

Liste todas las referencias en formato IEEE (artículos, libros, repositorios de datos o código, documentación de librerías, etc.).

\section{Anexo: Sesiones con herramientas de IA}

Incluya un anexo donde se documenten las interacciones con herramientas de IA generativa (por ejemplo, ChatGPT):

\begin{itemize}
  \item Preguntas realizadas.
  \item Respuestas obtenidas.
  \item Comentario crítico: qué partes de las respuestas se utilizaron y cuáles se descartaron.
  \item Cómo influyeron estas herramientas en la definición del problema y en los modelos de análisis.
\end{itemize}

\section*{Recomendaciones finales}

\begin{itemize}
  \item Mantenga clara la diferencia entre: \textbf{análisis del problema} y \textbf{diseño/implementación de la solución}.
  \item Verifique siempre que cada sección mencione explícitamente al menos una de las tres dimensiones (HP, CV, AI/ML/DL). Idealmente, conecte las tres.
  \item Utilice diagramas y tablas claros, con buena rotulación y descripciones breves.
  \item Priorice el uso de datos de contexto local (Ecuador o Latinoamérica) cuando sea posible.
\end{itemize}

\end{document}